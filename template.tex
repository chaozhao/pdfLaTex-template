\documentclass[11pt,a4paper]{report}
\usepackage[margin=2.54cm]{geometry}
\usepackage{hyperref}
\usepackage[T1]{fontenc}
\usepackage{colortbl}
\usepackage{color}
\usepackage{fancyhdr}
\usepackage[titles]{tocloft}
\usepackage{setspace}
\usepackage{array}
\hypersetup{colorlinks=true, linkcolor=blue, urlcolor=cyan}
\usepackage{pdflscape}
\usepackage[pdftex]{graphicx}
\usepackage[parfill]{parskip}
\usepackage{longtable}
\usepackage{float}
\usepackage[english]{babel}

%----------Config Data format---------------------------------------------
\usepackage{datetime}
\newdateformat{dashdate}{\THEYEAR-\twodigit{\THEMONTH}-\twodigit{\THEDAY}}
%-------------------------------------------------------------------------

\begin{document}
\raggedright{} %Left justified, make report more readable
\pagenumbering{roman}

%----------------------------Cover Page---------------------------------------
\title{This is report title}
\author{Author Name}
\date{today}
\maketitle

%--------------------------Revision History-----------------------------------
\renewcommand*\thesection{\Large{\arabic{section}}}
%\renewcommand*\thesubsection{\Large{\arabic{section. subsection}}}
\section*{Revision History}
	\begin{table}[h!]
		\centering
		\begin{longtable}{|c|c|c|p{7cm}|} \hline
%			\rowcolor[rgb]{0.4,0.8,0.9}  
			\textbf{Version} & \textbf{Date} & \textbf{Reviewer} & \textbf{Brief Details} \\\hline
			\endhead
			\endfoot
			\endlastfoot
			& & & \\\hline
			& & & \\\hline
			& & & \\\hline
			& & & \\\hline
			& & & \\\hline
			& & & \\\hline
			& & & \\\hline
			& & & \\\hline
			& & & \\\hline
			& & & \\\hline
			& & & \\\hline
			& & & \\\hline
			& & & \\\hline
			& & & \\\hline
			& & & \\\hline
		\end{longtable}
		%\caption{Revision History}
	\end{table}
\pagebreak

%----------------------------Executive Summary-----------------------------
\renewcommand{\abstractname}{Executive Summary}
\begin{abstract}
This is a abstraction

\end{abstract}


%----------------------------TABLE OF CONTENTS---------------------------------
\thispagestyle{empty}
\renewcommand\contentsname{Table of Contents}\tableofcontents
\listoffigures
\pagebreak

%----------------Footer/Header FOR REST OF DOCUMENT------------------------
\pagestyle{fancy}
\lhead{University}
\lfoot{Report Name}
\cfoot{\thepage}
\rfoot{Author Name}
\renewcommand{\headrulewidth}{0.4pt}
\renewcommand{\footrulewidth}{0.4pt}

%----------------Body of Document------------------------
\pagenumbering{arabic}

\section{Introduction}

\subsection{Purpose}

\subsection{Project Background}

\subsection{Scope}

\section{System Stakeholders and Concerns}

\section{System Architecture}
	
\subsection{Client--Server Description}

	\begin{figure}[h!]
		\centering
		\includegraphics[width=0.8\textwidth]{ClientServer.png}
		\caption{Client-server architecture in the context of VCE}
	\end{figure}
	
\subsection{Three-Tier Description}

	\begin{figure}[h!]
		\centering
		\includegraphics[width=0.8\textwidth]{ThreeTier.png}
		\caption{Three-tier architecture in the context of VCE}
	\end{figure}

\subsection{Architectural Views}

\subsubsection{Logical View}

\subsubsection{Physical View}

\subsection{Design Rational}

    The rational for choosing the proposed architecture can be traced to the
    fact that the VCE is an extension to a pre existing software package, known
    as the VRE and in order to maintain compatibility a similar design was
    chosen.  A military training tool known as VBS2 is a required interface to
    our product.  This led to the decision to use a client server architecture.


\end{document}
